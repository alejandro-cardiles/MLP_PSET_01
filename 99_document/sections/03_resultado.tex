\section{Interpretación de resultados}

Como se comentó previamente, el cambio estructural de la primera época concentró el desarrollo económico en las regiones del norte del país. 
Este patrón persiste hasta el presente. La figura \ref{fig:italy_map} muestra que la mayoría de las regiones del norte se ubican en los estratos más altos de productividad. 
En promedio, estas regiones presentan un nivel de producción por hora trabajadada 19\% superior al observado en el sur. Esta brecha sugiere que la divergencia 
agregada de Italia frente a otras economías europeas esta parcialmente explicada por diferencias regionales.

\begin{figure}[H]
    \centering
    \caption{Productividad laboral regional}
    \includegraphics[width=0.53\textwidth]{figures/03_mapa_productividad_italia.png}
    \label{fig:italy_map}

    \vspace{0.3em}
    {\footnotesize Nota: Agrupación del PIB por hora trabajada a precios constantes del 2020.}
    {\footnotesize Fuente: Eurostat.}
\end{figure}

Dado este patrón, se revisa nuevamente la producción por hora trabajada, desagregando el país en norte y sur. 
Como muestra la figura \ref{fig:north_south_vs_countries}, cuando se considera únicamente el norte, la brecha frente a Alemania se reduce a 7.9\% y 
frente a Francia a 1.01\% en 2019. En contraste, las regiones del sur exhiben diferencias mucho mayores: 
34.6\% frente a Alemania y 26.9\% frente a Francia. Esto indica que la divergencia internacional de Italia no es homogénea, 
sino que se encuentra concentrada territorialmente.

\begin{figure}[H]
    \centering
    \caption{Productividad laboral: comparación entre Italia Norte y Sur y países europeos}
    \label{fig:north_south_vs_countries}
    \includegraphics[width=0.8\textwidth]{figures/04_italia_norte_vs_sur-paises.png}
\end{figure}

Las diferencias entre estas dos regiones no solo recaen en la producción por hora, sino también en la estructura productiva. 
A pesar de que, en promedio, las personas en el sur de Italia trabajan un 4\% más de horas por año desde 2005, la producción por hora es aproximadamente 15\% 
menor que en el norte. Esto sugiere que la brecha no proviene de menor esfuerzo laboral, sino de diferencias en acumulación de capital y eficiencia productiva. 

En torno al capital, el sur ha mantenido un nivel por trabajador entre 73\% y 86\% del observado en el norte. 
Esta menor intensidad de capital indica que una parte de la variación entre el norte y el sur proviene de diferencias en la cantidad de capital por hora trabajada. 
Hay dos escenarios que son importantes en mencionar. La disminución debido a la crisis financiera de 2007, la cual redujo en un 2\% el capital con respecto al año previo observado en el sur,
y la crisis del euro, que disminuyó en un 4.8\% con respecto al norte. El decenso de este segundo caso se debe, en su mayoría, a un aumento en la inversión del norte de Italia con respecto al sur.

Por otro lado, el componente tecnológico muestra una disminución breve en el tiempo, descendiendo de 86\% a 83\% del nivel del norte en dos décadas. Exponiendo 
la idea de que la diferencia entre la producción recae en la falta de capital. 

\begin{figure}[H]
    \centering
    \caption{Brecha Sur/Norte por componentes: productividad, capital y tecnología}
    \label{fig:descomposition}
    \includegraphics[width=0.8\textwidth]{figures/05_italia_norte_vs_sur.png}
\end{figure}

En conclusión, la disparidad entre los niveles de producción de Italia y países similares se explica, en su mayoría, por una divergencia entre la producción del norte y la producción del sur. 
La cual se creó desde la década de los 50, en la cual una migración de trabajadores a industrias manufactureras generó un milagro económico. Al comparar la producción de estas dos regiones se observa que 
el sur mantiene un nivel inferior de inversión que el norte y las industrias de esta región parecen ser menos productivas que sus contrafactuales en el norte.
