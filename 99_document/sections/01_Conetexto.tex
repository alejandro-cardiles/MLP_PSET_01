\section{Contexto}
    
Esta sección presenta un ejercicio preliminar de detección de cobertura forrajera como aproximación operativa para caracterizar prácticas asociadas a sistemas con componente silvopastoril. El análisis combina datos de referencia provenientes de parcelas demostrativas del CIAT con imágenes satelitales PlanetScope de 3 m, a partir de las cuales se construye un set de entrenamiento a nivel de píxel y se ajusta un modelo de clasificación supervisada. La evaluación se realiza mediante validación espacial, reservando una zona completa como conjunto hold-out. En términos de desempeño, el modelo alcanzó un F1 de 0.53 en la zona fuera de muestra, con recall alto para la clase no forraje (0.96) y limitaciones asociadas a la identificación de forraje, en parte atribuibles a una posible delimitación parcial de los polígonos de referencia observada al contrastar con las predicciones espaciales. 