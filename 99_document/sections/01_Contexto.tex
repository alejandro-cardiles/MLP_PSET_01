 \section{Motivación y contexto}
    
El crecimiento de la producción italiana puede dividirse en tres períodos. 
El primero corresponde al rápido crecimiento de la posguerra, durante el cual se logró sextuplicar 
la producción per cápita en apenas treinta años. A este le sigue un período de crecimiento constante,
aunque inferior al observado previamente, en el que la productividad por hora trabajada aumentó aproximadamente un 56.8\%
en veinte años. Por último, la era actual, en la cual se observa un desacoplamiento con respecto a otras naciones europeas con un aumento solo
del 22.8\%. 

La primera época corresponde al periodo de 1950 a 1980 donde el  "milagro italiano" provino de un cambio estructural en 
el cual la masa laboral emigró a industrias manufactureras y a servicios modernos, aunque el cambio no fue homogéneo y
principalmente se concentró en el norte y algunas partes centrales del país \citep{structural_change_growth_2022}. 

La segunda época nace justo después de la crisis petrolera de los setenta, en la cual el mercado entra en una etapa 
de fluctuaciones en sus precios, caracterizada por etapas de fuertes aumentos en períodos cortos \citep{blanchard_gali_2010_oil}. Como se puede observar
en la figura \ref{fig:descomposition_county_level}, el impacto no fue diferencial para Italia debido ya que siguió creciendo a la par de otros países del 
continente europeo. 


\begin{figure}[H]
    \centering
    \caption{PIB por hora trabajada en Europa, 1950–2025}
    \label{fig:descomposition_county_level}
    \includegraphics[width=0.8\textwidth]{figures/01_productividad_por_trabajador.png}
\end{figure}

Finalmente, durante la era de los 2000 hasta el presente, se observa una separación con respecto a sus pares Alemania y Francia, y un acercamiento
a la producción de España. En otras palabras, mientras que la producción Italiana superaba a la Alemana por 20\% en 1980 y era inferior a la Francesa por solo 
5\% en 1995; ahora, en 2019, ambas economías sobrepasan los valores italianos por un valor alrededor del 20\%. En este contexto, se plantea la pregunta: ¿este nivel de separación
 en homogeneo en el territorio italiano, o surge del cambio estructural de los 50?