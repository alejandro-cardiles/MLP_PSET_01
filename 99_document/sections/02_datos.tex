\section{Datos}


%-------------------%
% descomposicion de la produción
%-------------------%
Para responder a esta diferencia, se realiza un ejercicio de contabilidad del crecimiento, donde se asume que la producción 
por hora depende del capital per capita $\frac{K}{L}$ multiplicado por un factor tecnológico $A$ (equación \ref{eq:productividad_per_capita}). 

\begin{equation}
    \frac{Y}{L} = A\left(\frac{K}{L}\right)^\alpha
    \label{eq:productividad_per_capita}
\end{equation}

%-------------------%
% fuentes
%-------------------%

Para la descomposición de la productividad se extrajo información de dos diferentes fuentes. La primera, Penn World Table (PWT) \citep{feenstra2024pwt},
contiene series historicas de variables económicas como población, producción, empleo y capital, entre otros. En su mayoría, la información que proviene de esta fuente, se usa para
realizar análisis de largo plazo, esto se debe a su cobertura histórica, que se extiende desde 1950. 

Por otro lado, se descargó información de la oficina estadística de la Unión Europea, EUROSTAT \citep{eurostat2024}. 
Esta base de datos cubre un período temporal más corto (a partir de 2000), pero ofrece un nivel de granularidad no disponible en la PWT. 
Gracias a ello, es posible acceder a información a nivel de la Nomenclatura de Unidades Territoriales para Estadísticas (NUTS, por sus siglas en inglés).

De ambas bases de datos se extrajeron valores para la producción $Y$, horas trabajadas $L$ y de capital $K$ solo se extrajeron valores de EUROSTAT. 
Asimismo, se aclara que el valor de productividad no es directamente comparable entre bases. 
Por ejemplo, la producción que proviene de la PWT se mide en valores de PPP\footnote{Valores en precios internacionales del 2021.} 
(unidad utilizada por economistas para comparar bienes sin preocuparse por las diferencias en niveles de precios entre países), 
mientras que EUROSTAT reporta los valores en PPS\footnote{Valores en precios europeos del 2020.}

Por último, para la medición de $L$ se sigue la metodología usada por \citep{gollin2002}, la cual mide el trabajo como el total de horas trabajadas. 
Además, el valor del capital $K$ se mide como la inversión total en valores nominales, la cual se ajusta a valores reales por medio de la división con el deflactor de nivel nacional.

