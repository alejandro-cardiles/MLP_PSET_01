\section{Resultados}

\subsection{Resultados preliminares}

El modelo entrenado obtuvo un F1 de 0.53 en la evaluación del conjunto hold-out (zona C). En este ejercicio, el F1 resume el desempeño del modelo para identificar píxeles con cobertura forrajera a partir de la información espectral de PlanetScope, al combinar (i) el recall de forraje, entendido como la proporción de píxeles forrajeros dentro de los polígonos de referencia que el modelo logra recuperar, y (ii) la precisión, definida como la proporción de píxeles predichos como forraje que efectivamente corresponden a forraje según los datos de referencia. En consecuencia, un F1 de 0.53 sugiere un desempeño intermedio: el modelo captura parcialmente la señal asociada a forraje, pero aún presenta errores en la identificación de píxeles forrajeros que reducen la confiabilidad de las predicciones de forraje en una zona no utilizada durante el entrenamiento. 

 

La Tabla \ref{tab:confusion_matrix_normalized} presenta la matriz de confusión para la zona C, que resume, a nivel de píxel, la correspondencia entre la clase asignada por los datos de referencia (polígonos de forraje) y la clase predicha por el modelo. En las filas se reporta la clase observada (no forraje y forraje) y en las columnas la clase predicha. Los valores en la diagonal representan aciertos, mientras que los valores fuera de la diagonal corresponden a errores de clasificación. En particular, el modelo muestra un desempeño alto para la clase no forraje: el recall de no forraje es 0.96, lo que indica que el 96\% de los píxeles etiquetados como no forraje en los datos de referencia fue correctamente identificados como tales. En contraste, el desempeño para la clase forraje se ve afectado principalmente por una precisión limitada: una proporción relevante de los píxeles que el modelo predice como forraje se ubica fuera de los polígonos de referencia y, por tanto, se contabiliza como no forraje en los datos de terreno. Este patrón incrementa los falsos positivos para la clase forraje y contribuye a reducir el F1 observado en la zona C. 

\begin{table}[htbp]
    \centering
    \caption{Matriz de confusión zona C}
    \label{tab:confusion_matrix_normalized}
    \input{visuals/tables/c_matrix_normalized}
\end{table}



Sin embargo, al contrastar las predicciones con la Figura \ref{fig:pred_forraje}, se identifica una explicación plausible para el patrón observado en la matriz de confusión. La figura sugiere que el modelo asigna la clase forraje en áreas espacialmente coherentes desde el punto de vista espectral, pero que no están totalmente cubiertas por los polígonos reportados por CIAT como referencia. Esto es consistente con la hipótesis de que el ground truth (pasturas reportadas por CIAT) presenta una delimitación parcial en algunas observaciones: parte de la cobertura forrajera presente en campo no habría quedado incluida en los polígonos y, por tanto, figura como no forraje en los datos de referencia. En ese escenario, predicciones plausiblemente correctas se contabilizan como falsos positivos al compararlas con etiquetas subdelimitadas, lo que reduce la precisión estimada para la clase forraje y contribuye a disminuir el F1 en la zona C. 

\begin{figure}[H]
    \centering
    \caption{Predicción de Forraje}
    \label{fig:pred_forraje}

    \begin{subfigure}[t]{0.45\textwidth}
        \centering
        \includegraphics[width=\textwidth]{visuals/img/pred.png}
        \caption{Predicción}
        \label{fig:pred_forraje_a}
    \end{subfigure}
    \hfill
    \begin{subfigure}[t]{0.45\textwidth}
        \centering
        \includegraphics[width=\textwidth]{visuals/img/truth.png}
        \caption{Pasturas reportadas por CIAT}
        \label{fig:pred_forraje_b}
    \end{subfigure}
\end{figure}


Adicionalmente, cuando se introduce una corrección que incorpora las áreas que, con base en la coherencia espectral y el contexto espacial, se consideran plausiblemente forrajeras aunque no estén incluidas en los polígonos de referencia del CIAT, el desempeño del modelo mejora. Bajo este escenario corregido, el valor del F1 para la clase forraje en la zona C se incrementa hasta 0.71, lo que indica que una fracción de los errores previamente contabilizados como falsos positivos corresponde en realidad a predicciones consistentes con la cobertura observada en campo.