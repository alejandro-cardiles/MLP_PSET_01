 \documentclass[11pt,onecolumn]{article}

% -------------------------------------------------
% Page layout
% -------------------------------------------------
\usepackage[margin=1.5in]{geometry}

% -------------------------------------------------
% Fonts and encoding
% -------------------------------------------------
\usepackage[T1]{fontenc}
\usepackage{times}

% -------------------------------------------------
% Math
% -------------------------------------------------
\usepackage{amsmath, amsfonts, amsthm, bbm}

% -------------------------------------------------
% Tables
% -------------------------------------------------
\usepackage{booktabs}
\usepackage{threeparttable}
\usepackage{longtable}
\usepackage{array}
\usepackage{dcolumn}
\usepackage[table]{xcolor}

\newcolumntype{L}[1]{>{\raggedright\arraybackslash}p{#1}}
\newcolumntype{C}[1]{>{\centering\arraybackslash}p{#1}}
\newcolumntype{R}[1]{>{\raggedleft\arraybackslash}p{#1}}

% -------------------------------------------------
% Figures
% -------------------------------------------------
\usepackage{graphicx}
\usepackage[labelsep=period]{caption}
\usepackage{subcaption}
\usepackage{float}
\usepackage{rotating}
%\usepackage{rotfloat}
\usepackage{wrapfig}

% -------------------------------------------------
% Spacing and structure
% -------------------------------------------------
\usepackage{setspace}
\usepackage{titlesec}
\titlelabel{\thetitle.\quad}
\renewcommand{\baselinestretch}{1}

% -------------------------------------------------
% Footnotes
% -------------------------------------------------
\usepackage[bottom]{footmisc}
\setlength\footnotemargin{5pt}

% -------------------------------------------------
% Citations
% -------------------------------------------------
\usepackage{natbib}
\bibliographystyle{apalike}
\bibpunct{(}{)}{;}{a}{}{,}

% -------------------------------------------------
% Utility packages
% -------------------------------------------------
\usepackage{placeins}
\usepackage{comment}
\usepackage{soul}
\usepackage{xspace}
\newcommand\nth{\textsuperscript{th}\xspace}

% -------------------------------------------------
% Landscape
% -------------------------------------------------
\usepackage{lscape}
\usepackage{pdflscape}
\usepackage{multicol}

% -------------------------------------------------
% Language (Spanish captions: Figura, Tabla, etc.)
% -------------------------------------------------
\renewcommand{\figurename}{Figura}
\renewcommand{\tablename}{Tabla}


% -------------------------------------------------
% TikZ
% -------------------------------------------------
\usepackage{tikz}
\usetikzlibrary{
  decorations,
  decorations.text,
  shapes.geometric,
  arrows.meta,
  automata
}

\definecolor{timeline2015}{RGB}{212,212,212}
\definecolor{timeline2016}{RGB}{48,201,204}
\definecolor{timeline2017}{RGB}{45,87,204}
\definecolor{timeline2018}{RGB}{162,45,204}
\definecolor{timeline2019}{RGB}{82,45,204}

% -------------------------------------------------
% Theorem environments
% -------------------------------------------------
\theoremstyle{plain}
\newtheorem{proposition}{Proposition}
\newtheorem{theorem}{Theorem}
\newtheorem{lemma}{Lemma}
\newtheorem{corollary}{Corollary}

\theoremstyle{definition}
\newtheorem{definition}{Definition}
\newtheorem{example}{Example}
\newtheorem{assumption}{Assumption}

\theoremstyle{remark}
\newtheorem{remark}{Remark}
\newtheorem{note}{Note}

% -------------------------------------------------
% Hyperref (LAST)
% -------------------------------------------------
\usepackage[
  colorlinks=true,
  linkcolor=black,
  urlcolor=blue,
  citecolor=blue
]{hyperref}
\def\UrlBreaks{\do\/\do-}

% -------------------------------------------------
% Title and authors
% -------------------------------------------------
\title{Italia: La divergencia entre dos naciones}
%\thanks{We thank ... All errors are our own.}}

\author{
Alejandro Cardiles\thanks{Department of Economics, University of Los Andes, Colombia. \href{mailto:a.cardilesh@uniandes.edu.co}{a.cardilesh@uniandes.edu.co}} 
\thanks{Información y códigos disponibles para replicabilidad pueden ser encontrados en: \href{ttps://github.com/alejandro-cardiles/MLP_PSET_01.git}{GITHUB}}
}

\date{\today}

% -------------------------------------------------
\begin{document}
% -------------------------------------------------

\maketitle
\thispagestyle{empty}

%\begin{abstract}
%\noindent
%\end{abstract}

\medskip
\begin{flushleft}
%\textbf{Keywords:} \\
%\textbf{JEL Classification:}
\end{flushleft}

\pagebreak
\doublespacing

%------------------------%
% Sections
%-----------------------%

\input{sections/01_contexto}
\section{Datos}


\subsection{Datos de referencia}

Para el componente silvopastoril, la información de referencia provino de polígonos geolocalizados de parcelas demostrativas del CIAT conteniendo pasturas forrajeras ubicadas en tres zonas del país. La figura \ref{fig:par_demostrativas} muestra la ubicación de estas parcelas que para efectos del análisis, fueron denominadas A, B, C y D. Por su ubicación, consideramos las parcelas B y D parte de una misma zona ya que se encuentran separadas por apenas 60 metros. En contraste, la parcela C se ubica a 46,45 km de ese par, mientras que la parcela A es la más distante, localizada a 234,60 km. 

\begin{figure}[H]
    \centering
    \caption{Parcelas demostrativas del CIAT con pasturas forrajeras}
    \includegraphics[width=0.5\textwidth]{visuals/img/location_plots.png}
    \label{fig:par_demostrativas}
\end{figure}


%%%----- imagen 

Estas parcelas demostrativas fueron establecidas en momentos distintos, por lo que los datos de referencia corresponden únicamente a las fechas en las que la cobertura forrajera era detectable. La Tabla \ref{tab:forraje_planet} presenta (columna 2) las fechas para las cuales se tiene certeza de la presencia de forraje en cada parcela: la observación más antigua corresponde a la parcela D (2022-07-22) y la más reciente a la parcela A (2025-02-28). Adicionalmente, la disponibilidad de observaciones no es homogénea: las parcelas A y D cuentan con tres observaciones temporales, mientras que las parcelas B y C disponen de una observación cada una. 



\begin{table}[htbp]
    \centering
    \caption{Conteo y proporción de forraje por fecha}
    \label{tab:forraje_planet}
    \input{visuals/tables/table_planet_download_time} 
\end{table}





\subsection{Datos Satelitales}

Para estas primeras aproximaciones se emplearon imágenes PlanetScope (producto Surface Reflectance), con cobertura diaria y resolución espacial de 3 m. Cada escena incluye ocho bandas espectrales (Coastal Blue, Blue, Green I, Green, Yellow, Red, Red Edge y NIR), las cuales se utilizaron como insumo para derivar índices espectrales de vegetación asociados a vigor y productividad de la cobertura. 

 
Dado que, para el componente silvopastoril/forrajes, no se dispone de una serie temporal continua perfectamente alineada con las fechas de certeza en terreno, se seleccionó para cada parcela la escena de PlanetScope más cercana a la fecha en la que se tiene confirmación de presencia de forraje, priorizando escenas con baja interferencia por nubes y sombras. La Tabla \ref{tab:forraje_planet} consolida esta correspondencia: la columna (3) reporta el día de la escena PlanetScope utilizada; la columna (4) indica la diferencia en días entre la fecha de certeza del dato de referencia (columna (2)) y la fecha de la escena PlanetScope (columna (3)), conservando el signo para reflejar si la imagen es anterior o posterior a la fecha de certeza. Esta selección se realizó para evitar obstrucciones como nubes o sombras. Finalmente, la columna (5) presenta el número de píxeles que corresponden a las parcelas con forrajes de la escena analizada, mientras que la columna (6) reporta el número de píxeles sin forraje en esa misma escena. 


\subsection{Preparación de datos para predicción }

Con el fin de asegurar consistencia entre las etiquetas derivadas de los datos de referencia y los predictores espectrales de PlanetScope, el set de entrenamiento se construyó a nivel de píxel para cada una de las escenas seleccionadas en la Tabla \ref{tab:forraje_planet}. En primer lugar, los polígonos georreferenciados de las parcelas demostrativas se proyectaron sobre la grilla de 3 m de PlanetScope y se rasterizaron para identificar los píxeles ubicados dentro de la cobertura forrajera (clase forraje) y los píxeles ubicados fuera de dicha cobertura (clase no forraje) dentro del recorte analizado. 

 

Para reducir la mezcla de clases por efectos de borde y minimizar la contaminación espacial entre coberturas adyacentes, se aplicó un buffer de 10–20 m alrededor de los límites de las coberturas de interés. Este ajuste buscó excluir píxeles en zonas de transición donde la señal espectral puede representar combinaciones de forraje y coberturas vecinas, y donde la asignación de clase es más incierta. 

 

A partir de los píxeles retenidos tras el buffer, se extrajeron como variables predictoras las bandas espectrales y los índices de vegetación derivados de cada escena donde se calcularon cuatro índices (NDVI, EVI, FAPAR y NIRv), siguiendo la metodología propuesta por \textcolor{red}{Referencia}.

\begin{figure}[H]
    \centering
    \caption{Índices espaciales de parcelas demostrativas }
    \includegraphics[width=0.9\textwidth]{visuals/img/index.png}
    \label{fig:index_satelite}
\end{figure}

  
\section{Interpretación de resultados}

Como se comentó previamente, el cambio estructural de la primera época concentró el desarrollo económico en las regiones del norte del país. 
Este patrón persiste hasta el presente. La figura \ref{fig:italy_map} muestra que la mayoría de las regiones del norte se ubican en los estratos más altos de productividad. 
En promedio, estas regiones presentan un nivel de producción por hora trabajadada 19\% superior al observado en el sur. Esta brecha sugiere que la divergencia 
agregada de Italia frente a otras economías europeas esta parcialmente explicada por diferencias regionales.

\begin{figure}[H]
    \centering
    \caption{Productividad laboral regional}
    \includegraphics[width=0.53\textwidth]{figures/03_mapa_productividad_italia.png}
    \label{fig:italy_map}

    \vspace{0.3em}
    {\footnotesize Nota: Agrupación del PIB por hora trabajada a precios constantes del 2020.}
    {\footnotesize Fuente: Eurostat.}
\end{figure}

Dado este patrón, se revisa nuevamente la producción por hora trabajada, desagregando el país en norte y sur. 
Como muestra la figura \ref{fig:north_south_vs_countries}, cuando se considera únicamente el norte, la brecha frente a Alemania se reduce a 7.9\% y 
frente a Francia a 1.01\% en 2019. En contraste, las regiones del sur exhiben diferencias mucho mayores: 
34.6\% frente a Alemania y 26.9\% frente a Francia. Esto indica que la divergencia internacional de Italia no es homogénea, 
sino que se encuentra concentrada territorialmente.

\begin{figure}[H]
    \centering
    \caption{Productividad laboral: comparación entre Italia Norte y Sur y países europeos}
    \label{fig:north_south_vs_countries}
    \includegraphics[width=0.8\textwidth]{figures/04_italia_norte_vs_sur-paises.png}
\end{figure}

Las diferencias entre estas dos regiones no solo recaen en la producción por hora, sino también en la estructura productiva. 
A pesar de que, en promedio, las personas en el sur de Italia trabajan un 4\% más de horas por año desde 2005, la producción por hora es aproximadamente 15\% 
menor que en el norte. Esto sugiere que la brecha no proviene de menor esfuerzo laboral, sino de diferencias en acumulación de capital y eficiencia productiva. 

En torno al capital, el sur ha mantenido un nivel por trabajador entre 73\% y 86\% del observado en el norte. 
Esta menor intensidad de capital indica que una parte de la variación entre el norte y el sur proviene de diferencias en la cantidad de capital por hora trabajada. 
Hay dos escenarios que son importantes en mencionar. La disminución debido a la crisis financiera de 2007, la cual redujo en un 2\% el capital con respecto al año previo observado en el sur,
y la crisis del euro, que disminuyó en un 4.8\% con respecto al norte. El decenso de este segundo caso se debe, en su mayoría, a un aumento en la inversión del norte de Italia con respecto al sur.

Por otro lado, el componente tecnológico muestra una disminución breve en el tiempo, descendiendo de 86\% a 83\% del nivel del norte en dos décadas. Exponiendo 
la idea de que la diferencia entre la producción recae en la falta de capital. 

\begin{figure}[H]
    \centering
    \caption{Brecha Sur/Norte por componentes: productividad, capital y tecnología}
    \label{fig:descomposition}
    \includegraphics[width=0.8\textwidth]{figures/05_italia_norte_vs_sur.png}
\end{figure}

En conclusión, la disparidad entre los niveles de producción de Italia y países similares se explica, en su mayoría, por una divergencia entre la producción del norte y la producción del sur. 
La cual se creó desde la década de los 50, en la cual una migración de trabajadores a industrias manufactureras generó un milagro económico. Al comparar la producción de estas dos regiones se observa que 
el sur mantiene un nivel inferior de inversión que el norte y las industrias de esta región parecen ser menos productivas que sus contrafactuales en el norte.

\bibliography{sections/references}

\end{document}
